% IMU线速度估算技术文档 - LaTeX版本
\documentclass[12pt,a4paper]{article}

% 使用适合Overleaf的中文设置
\usepackage[UTF8, scheme=plain]{ctex}
% 或者使用xeCJK代替ctex
% \usepackage{xeCJK}
% \setCJKmainfont{SimSun} % 使用宋体

\usepackage{amsmath,amssymb,amsfonts} % 数学符号
\usepackage{graphicx} % 图片
\usepackage{booktabs} % 表格
\usepackage{listings} % 代码
\usepackage{xcolor} % 颜色
\usepackage{hyperref} % 超链接

% 修改代码样式设置,避免特殊字符问题
\lstset{
    basicstyle=\ttfamily\small,
    keywordstyle=\color{blue},
    commentstyle=\color{green!60!black},
    stringstyle=\color{red},
    breaklines=true,
    breakatwhitespace=true,
    frame=single,
    showstringspaces=false,
    upquote=true,
    extendedchars=true,
    literate={*}{{\textasteriskcentered}}1
}

\title{IMU线速度估算技术文档}
\author{}
\date{}

\begin{document}

\maketitle

\section{基本原理}

线速度估算是机器人导航和控制的关键环节。在没有直接速度传感器的情况下,基于IMU数据的速度估算提供了一种可行方案。本文档详细介绍通过IMU加速度数据估算线速度的方法、误差控制策略和数学原理。

\subsection{问题定义}

在强化学习环境中,机器人观测空间包含线速度信息。然而,在实际硬件上,我们通常只有IMU提供的加速度和姿态数据,需要通过计算推导出线速度。

\subsection{核心思路}

线速度估算基于物理学基本原理:速度是加速度的积分。因此,我们可以将IMU测量的加速度进行积分,得到速度变化。然而,这一过程存在多种误差来源,需要应用各种技术来减少累积误差。

\section{数学基础}

\subsection{四元数表示与旋转}

四元数是表示3D旋转的高效方法,比欧拉角更稳定,比旋转矩阵更紧凑。

\subsubsection{四元数定义}

四元数 $q = [q_0, q_1, q_2, q_3] = [w, x, y, z]$ 其中:
\begin{itemize}
    \item $w$: 实部
    \item $x, y, z$: 虚部
\end{itemize}

单位四元数表示旋转:$w^2 + x^2 + y^2 + z^2 = 1$

\subsubsection{四元数旋转公式}

使用四元数对向量 $v$ 进行旋转的公式:

$v' = q \otimes v \otimes q^{-1}$

其中 $\otimes$ 表示四元数乘法,$q^{-1}$ 是 $q$ 的逆四元数。

在代码中,这一旋转的具体实现为:

\begin{lstlisting}
def quat_rotate(q, v):
    w, x, y, z = q
    vx, vy, vz = v
    
    result = np.zeros(3)
    result[0] = (1 - 2*y*y - 2*z*z)*vx + 2*(x*y - w*z)*vy + 2*(x*z + w*y)*vz
    result[1] = 2*(x*y + w*z)*vx + (1 - 2*x*x - 2*z*z)*vy + 2*(y*z - w*x)*vz
    result[2] = 2*(x*z - w*y)*vx + 2*(y*z + w*x)*vy + (1 - 2*x*x - 2*y*y)*vz
    
    return result
\end{lstlisting}

\subsection{积分方法}

\subsubsection{基本积分公式}

速度 $v$ 与加速度 $a$ 的关系:

$v(t) = v(t_0) + \int_{t_0}^{t} a(\tau) d\tau$

离散形式:

$v_k = v_{k-1} + a_k \cdot \Delta t$

\subsubsection{梯形积分法}

为提高精度,可使用梯形法:

$v_k = v_{k-1} + \frac{1}{2}(a_k + a_{k-1}) \cdot \Delta t$

\section{重力补偿}

\subsection{重力影响}

IMU测量的加速度包含惯性加速度和重力加速度的综合影响。我们需要移除重力分量,只保留运动产生的加速度。

\subsection{补偿方法}

我们采用"直接在世界坐标系中补偿重力"的方法,步骤如下:

\begin{enumerate}
    \item 获取IMU加速度数据和姿态四元数
    \item 使用四元数将IMU加速度转换到世界坐标系
    \item 在世界坐标系中直接减去重力向量
    \item 在世界坐标系中进行后续计算
\end{enumerate}

\begin{lstlisting}
# 获取IMU数据
linear_acc = np.array(sensors.get_linear_acceleration())  # IMU坐标系下的加速度
quaternion, _ = sensors.read_imu()
q0, q1, q2, q3 = quaternion  # 四元数[w, x, y, z]

# 使用四元数共轭进行反向旋转(从IMU坐标系到世界坐标系)
linear_acc_world = quat_rotate([q0, -q1, -q2, -q3], linear_acc)

# 在世界坐标系中减去重力
gravity_world = np.array([0, 0, 9.81])  # 世界坐标系下的重力向量
linear_acc_no_gravity = linear_acc_world - gravity_world
\end{lstlisting}

与传统的"在IMU坐标系中补偿重力"方法相比,这种方法更加简洁,避免了最后还需要将速度从IMU坐标系转换到世界坐标系的步骤。

\section{积分漂移问题与解决方案}

\subsection{漂移原因}

\begin{enumerate}
    \item \textbf{传感器噪声}:IMU数据存在随机噪声
    \item \textbf{采样误差}:离散采样导致积分误差
    \item \textbf{偏置误差}:IMU传感器存在零点漂移
    \item \textbf{积分累积}:小误差通过积分不断累加
\end{enumerate}

\subsection{解决方案}

\subsubsection{低通滤波}

使用低通滤波减少高频噪声:

$v_{\text{filtered}} = (1 - \alpha) \cdot v_{\text{previous}} + \alpha \cdot v_{\text{current}}$

其中 $\alpha$ 是滤波系数(0-1之间)。

\subsubsection{零速度检测与校正}

当检测到机器人可能静止时(加速度接近零),强制速度逐渐衰减至零:

\begin{lstlisting}
accel_magnitude = np.linalg.norm(linear_acc_no_gravity)
if accel_magnitude < threshold:  # 可能静止
    decay_factor = np.exp(-dt / time_constant)
    velocity *= decay_factor
\end{lstlisting}

\subsubsection{速度限幅}

防止速度估计发散,设定最大速度阈值:

\begin{lstlisting}
max_velocity = 2.0  # 最大速度限制
velocity_magnitude = np.linalg.norm(velocity)
if velocity_magnitude > max_velocity:
    velocity = velocity * (max_velocity / velocity_magnitude)
\end{lstlisting}

\subsubsection{互补滤波}

结合其他信息源(如视觉里程计、车轮编码器等)进行互补滤波。

\section{实现算法流程}

\subsection{完整算法步骤}

\begin{enumerate}
    \item \textbf{初始化}:设置初始速度和时间戳
    \item \textbf{数据获取}:读取IMU加速度和姿态数据
    \item \textbf{坐标转换}:将IMU加速度转换到世界坐标系
    \item \textbf{重力补偿}:在世界坐标系中减去重力
    \item \textbf{速度积分}:积分加速度得到速度增量
    \item \textbf{滤波处理}:应用低通滤波减少噪声
    \item \textbf{漂移校正}:检测静止状态并校正速度
    \item \textbf{限幅保护}:防止速度估计发散
\end{enumerate}

\subsection{核心代码实现}

\begin{lstlisting}
def estimate_linear_vel():
    global global_linear_vel, last_update_time
    
    # 计算时间增量
    current_time = time.time()
    dt = current_time - last_update_time
    
    # 防止dt过小或首次运行
    if dt < 0.001:
        return global_linear_vel.tolist()
    
    # 获取IMU数据
    linear_acc = np.array(sensors.get_linear_acceleration())  # IMU坐标系下的加速度
    quaternion, _ = sensors.read_imu()
    q0, q1, q2, q3 = quaternion  # 四元数[w, x, y, z]
    
    # 使用四元数共轭进行反向旋转(从IMU坐标系到世界坐标系)
    linear_acc_world = quat_rotate([q0, -q1, -q2, -q3], linear_acc)
    
    # 在世界坐标系中减去重力
    gravity_world = np.array([0, 0, 9.81])  # 重力加速度
    linear_acc_no_gravity = linear_acc_world - gravity_world
    
    # 积分计算速度变化
    delta_v = linear_acc_no_gravity * dt
    
    # 低通滤波
    alpha = 0.8
    global_linear_vel = (1 - alpha) * global_linear_vel + alpha * (global_linear_vel + delta_v)
    
    # 零速度校正
    accel_magnitude = np.linalg.norm(linear_acc_no_gravity)
    if accel_magnitude < 0.1:
        decay_factor = np.exp(-dt / 0.5)
        global_linear_vel *= decay_factor
    
    # 速度限幅
    max_velocity = 2.0
    velocity_magnitude = np.linalg.norm(global_linear_vel)
    if velocity_magnitude > max_velocity:
        global_linear_vel = global_linear_vel * (max_velocity / velocity_magnitude)
    
    last_update_time = current_time
    return global_linear_vel.tolist()  # 直接返回世界坐标系下的速度
\end{lstlisting}

\section{性能优化与参数调整}

\subsection{关键参数解释}

\begin{table}[htbp]
\centering
\begin{tabular}{@{}llll@{}}
\toprule
参数 & 描述 & 典型值 & 影响 \\
\midrule
alpha & 低通滤波系数 & 0.8 & 越大越信任当前测量,越小越平滑 \\
accel\_threshold & 静止检测阈值 & 0.1 m/s² & 越小越容易被判定为静止 \\
decay\_factor & 速度衰减系数 & exp(-dt/0.5) & 影响静止时速度归零速率 \\
max\_velocity & 速度上限 & 2.0 m/s & 限制最大估计速度 \\
\bottomrule
\end{tabular}
\caption{关键参数及其影响}
\end{table}

\subsection{调优建议}

\begin{enumerate}
    \item \textbf{增加预热阶段}:
    在开始控制前,让机器人保持静止并校准初始状态

    \item \textbf{环境适应性调整}:
    \begin{itemize}
        \item 平坦地面:可使用较大的alpha值
        \item 崎岖地形:降低alpha值,增加平滑性
    \end{itemize}
    
    \item \textbf{针对不同机器人调整}:
    \begin{itemize}
        \item 大型机器人:增大max\_velocity和降低accel\_threshold
        \item 小型机器人:减小max\_velocity和提高accel\_threshold
    \end{itemize}
\end{enumerate}

\section{高级扩展:卡尔曼滤波器}

当前实现使用的是简化的滤波方法。更高级的实现可以考虑卡尔曼滤波器。

\subsection{卡尔曼滤波原理}

卡尔曼滤波器通过预测-更新循环,结合系统模型和测量值,得到最优状态估计。

\subsection{应用到速度估计}

\begin{enumerate}
    \item \textbf{状态向量}:包含位置和速度
    \item \textbf{系统模型}:描述位置和速度的演化
    \item \textbf{测量模型}:加速度测量与状态的关系
    \item \textbf{噪声建模}:对系统噪声和测量噪声进行建模
\end{enumerate}

\subsection{Python实现示例}

\begin{lstlisting}
def kalman_filter_velocity_estimation():
    # 简化的卡尔曼滤波器实现
    # 需要更详细的状态模型和噪声参数
    pass
\end{lstlisting}

\section{常见问题与解决方案}

\subsection{速度漂移}

\textbf{症状}:静止时速度不为零,或速度持续增长\\
\textbf{解决}:增强零速度检测,调整decay\_factor

\subsection{过度平滑}

\textbf{症状}:速度响应滞后,无法反映快速变化\\
\textbf{解决}:增大alpha值,减小滤波强度

\subsection{噪声干扰}

\textbf{症状}:速度估计抖动明显\\
\textbf{解决}:减小alpha值,增强滤波效果

\section{结论与建议}

线速度估算是机器人控制系统的重要组成部分。通过合理的数学模型和滤波技术,可以从IMU数据中获得较为可靠的速度估计。由于积分漂移的客观存在,建议:

\begin{enumerate}
    \item 定期重置或校准速度估计
    \item 结合多种信息源(如有)进行融合
    \item 针对具体应用场景调整参数
    \item 考虑更高级的滤波方法(如卡尔曼滤波)
    \item 实时监测估计质量,发现异常及时干预
\end{enumerate}

通过以上技术和方法,可以有效提高线速度估算的准确性和稳定性,为机器人控制提供可靠的速度反馈。

\end{document} 